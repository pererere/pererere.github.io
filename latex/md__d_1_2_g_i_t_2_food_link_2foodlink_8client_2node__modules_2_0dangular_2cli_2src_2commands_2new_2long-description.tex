\chapter{long-\/description}
\hypertarget{md__d_1_2_g_i_t_2_food_link_2foodlink_8client_2node__modules_2_0dangular_2cli_2src_2commands_2new_2long-description}{}\label{md__d_1_2_g_i_t_2_food_link_2foodlink_8client_2node__modules_2_0dangular_2cli_2src_2commands_2new_2long-description}\index{long-\/description@{long-\/description}}
Creates and initializes a new Angular application that is the default project for a new workspace.

Provides interactive prompts for optional configuration, such as adding routing support. All prompts can safely be allowed to default.


\begin{DoxyItemize}
\item The new workspace folder is given the specified project name, and contains configuration files at the top level.
\item By default, the files for a new initial application (with the same name as the workspace) are placed in the {\ttfamily src/} subfolder.
\item The new application\textquotesingle{}s configuration appears in the {\ttfamily projects} section of the {\ttfamily angular.\+json} workspace configuration file, under its project name.
\item Subsequent applications that you generate in the workspace reside in the {\ttfamily projects/} subfolder.
\end{DoxyItemize}

If you plan to have multiple applications in the workspace, you can create an empty workspace by using the {\ttfamily -\/-\/no-\/create-\/application} option. You can then use {\ttfamily ng generate application} to create an initial application. This allows a workspace name different from the initial app name, and ensures that all applications reside in the {\ttfamily /projects} subfolder, matching the structure of the configuration file. 