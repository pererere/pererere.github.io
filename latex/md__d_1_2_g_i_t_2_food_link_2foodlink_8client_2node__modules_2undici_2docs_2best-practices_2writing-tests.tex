\chapter{Writing tests}
\hypertarget{md__d_1_2_g_i_t_2_food_link_2foodlink_8client_2node__modules_2undici_2docs_2best-practices_2writing-tests}{}\label{md__d_1_2_g_i_t_2_food_link_2foodlink_8client_2node__modules_2undici_2docs_2best-practices_2writing-tests}\index{Writing tests@{Writing tests}}
\label{md__d_1_2_g_i_t_2_food_link_2foodlink_8client_2node__modules_2undici_2docs_2best-practices_2writing-tests_autotoc_md21499}%
\Hypertarget{md__d_1_2_g_i_t_2_food_link_2foodlink_8client_2node__modules_2undici_2docs_2best-practices_2writing-tests_autotoc_md21499}%


Undici is tuned for a production use case and its default will keep a socket open for a few seconds after an HTTP request is completed to remove the overhead of opening up a new socket. These settings that makes Undici shine in production are not a good fit for using Undici in automated tests, as it will result in longer execution times.

The following are good defaults that will keep the socket open for only 10ms\+:


\begin{DoxyCode}{0}
\DoxyCodeLine{import\ \{\ request,\ setGlobalDispatcher,\ Agent\ \}\ from\ 'undici'}
\DoxyCodeLine{}
\DoxyCodeLine{const\ agent\ =\ new\ Agent(\{}
\DoxyCodeLine{\ \ keepAliveTimeout:\ 10,\ //\ milliseconds}
\DoxyCodeLine{\ \ keepAliveMaxTimeout:\ 10\ //\ milliseconds}
\DoxyCodeLine{\})}
\DoxyCodeLine{}
\DoxyCodeLine{setGlobalDispatcher(agent)}

\end{DoxyCode}
 